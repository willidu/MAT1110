\documentclass[a4paper,10pt,english]{article}
\usepackage[utf8]{inputenc}
\usepackage[english]{babel}
\usepackage{amsmath,graphicx,varioref,verbatim,amsfonts,geometry}
\usepackage[usenames,dvipsnames,svgnames,table]{xcolor}
\usepackage[colorlinks=false]{hyperref}
\usepackage{bm}

\setlength{\parindent}{0mm}
\setlength{\parskip}{1.5mm}

\usepackage{textcomp}
\definecolor{listinggray}{gray}{0.9}
\definecolor{lbcolor}{rgb}{0.9,0.9,0.9}

\usepackage{listings}

\lstdefinelanguage{python}
{
	morekeywords={print,abs,for,def,if,while,do,break,return,from,import,try,except,else,elif},
	sensitive=false,
	morecomment=[l]{\#}
}

\lstset{language=python,
	backgroundcolor=\color[rgb]{.95,.95,.95},
	numbers=left,xleftmargin=10pt,
	numberstyle=\tiny,stepnumber=1,numbersep=5pt,
	stringstyle=\color{red},
	basicstyle=\footnotesize \ttfamily,
	keywordstyle=\color{blue},
	commentstyle=\color{green},
	basewidth=0.60em,
	showstringspaces=false,
	captionpos=b,
	frame=single
}

% \renewcommand{\thesubsection}{\roman{subsection}}
% \renewcommand{\thesection}{\alph{section}}
\renewcommand{\thesubsection}{\alph{subsection}}

\title{MAT1110 - Mandatory assignment 2}
\author{William Dugan}

\begin{document}

\maketitle

\section{} \label{1}

\subsection{} \label{1a}

We have
\begin{align}
    \bm{F} (x, y) = - \frac{y}{2} \bm{i} + \frac{x}{2} \bm{j}
\end{align}
and the parameterization $\bm{r}$ of $C$ oriented counter-clockwise. If follows that $\bm{r}$ is piecewise smooth as $C$ is piecewise smooth. Since $C$ encloses an area including $R$, and the partial derivatives of $\bm{F}$ are continuous, we can use Greens' theorem two write the area enclosed by $C$ as
\begin{align*}
    \iint_R \left( \frac{\partial Q}{\partial x} - \frac{\partial P}{\partial y} \right) dA 
    = \oint_C \bm{F} \cdot d\bm{r}
\end{align*}

\subsection{} \label{1b}
Since $C_k$ is the line piece connecting the points $(a_k, b_k)$ and $(a_{k+1}, b_{k+1})$ can we write \newline $\Delta x = a_{k+1} - a_k$ and $\Delta y = b_{k+1} - b_k$ from the point $(a_k, b_k)$. The change will be linear as $C_k$ are straight lines. If we put all this together, we get the parameterization
\begin{align}
    &\bm{r}_k (t) = 
    (a_k + t (a_{k+1} - a_k)), b_k + t (b_{k+1} - b_k)), 
    & &t \in [0, 1]
\end{align}

\subsection{} \label{1c}
\begin{align*}
    A_k = \int_{C_k} \bm{F} \cdot d\bm{r} = \int_{C_k} x dy
\end{align*}
If we use $x = a_k + t (a_{k+1} - a_k)$ and $dx = ( b_{k+1} - b_k) dt$ we get
\begin{align*}
    A_k &= (b_{k+1} - b_k) \int_0^1 (a_k + t (a_{k+1} - a_k)) dt \\
    &= (b_{k+1} - b_k) \left[ ta_k + \frac{1}{2} t^2 (a_{k+1} - a_k) \right]_0^1 \\
    &= \frac{1}{2} (a_{k+1} + a_k)(b_{k+1} - b_k)
\end{align*}
If we sum over all line-pieces we get
\begin{align} \label{eq:sum}
    A = \frac{1}{2} \sum_{k=1}^{n-1} (a_{k+1} + a_k)(b_{k+1} - b_k)
\end{align}

\subsection{} \label{1d}
We will now calculate the area of a triangle with corners $(0, 0), (a, h)$ and $(g, 0)$ using equation \ref{eq:sum}.
\begin{align*}
    A_{\text{triangle}} &= \frac{1}{2} [(g-0)(0-0) + (a+g)(h-0) + (0+a)(0-h)] \\
    &= \frac{1}{2} [ah + gh - ah] \\
    &= \frac{gh}{2}
\end{align*}

For a rectangle with corners $(0, 0), (g, 0), (g, h)$ and $(0, h)$ we get
\begin{align*}
    A_{\text{rectangle}} &= \frac{1}{2} [(g+0)(0-0) + (g+g)(h-0) + (0+g)(h-h) + (0+0)(0-h)] \\
    &= \frac{1}{2} [2gh] \\
    &= gh.
\end{align*}

\section{} \label{2}

\newpage

\section{} \label{3}
We define $f: \mathbb{R}^2 \to \mathbb{R}^2$ where $f(\bm{v}) = A \bm{v} + \bm{b}$.

\subsection{} \label{3a}
We let $f$ be an isometry in the xy-plane. It is easy to show that $f$ preserves norms:
\begin{align*}
    ||\bm{v} - \bm{w}|| &= ||f(\bm{v}) - f(\bm{w}))|| \\
    &= ||A\bm{v} + \bm{b} - (A\bm{w} + \bm{b})|| \\
    &= ||A\bm{v} - A\bm{w}|| \\
    &= ||A|| \cdot ||\bm{v} - \bm{w}|| \\
    &= ||\bm{v} - \bm{w}||
\end{align*}
where we have used that det$(A)=\pm 1$.

\subsection{} \label{3b}
We let $f(\bm{v}) = A \bm{v} + \bm{b}$ be an isometry with det$(A)=-1$. The vector $A \bm{b} + \bm{b}$ is an eigenvector if $(A-I)(A \bm{b} + \bm{b}) = 0$ for all choises of $\bm{b}$. Remembering that $A^2 = I$, we get
\begin{align*}
    (A-I)(A \bm{b} + \bm{b}) &= A^2 \bm{b} + A \bm{b} - AI \bm{b} - I \bm{b} = 0
\end{align*}
which shows that $A \bm{b} + \bm{b}$ is an eigenvector.

\subsection{} \label{3c}
The transformation by $f$ on the line perpendicular to $\bm{w}$ is
\begin{align*}
    f(s\bm{w}) = s A \bm{w} + \bm{b}, \hspace{1cm} s \in \mathbb{R}
\end{align*}
Since $\bm{w}$ is an eigenvector $A \bm{w} = \lambda \bm{w}$. Remembering that $\lambda = 1$, we can then scale the transformation down by a half, and get
\begin{align} \label{eq:line}
    t \bm{w} + \frac{1}{2} \bm{b}, \hspace{1cm} t \in \mathbb{R}
\end{align}
where we let $t = s/2$. This shows that $f$ transforms the line in eq \ref{eq:line} on itself.

\end{document}
